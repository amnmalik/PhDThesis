\documentclass[../thesis.tex]{subfiles}
\graphicspath{{\subfix{../pictures/}}}
\begin{document}
\chapter{Synthesis and Outlook}\label{ch:summary}
\section{Summary of chapters}
The three studies presented in the preceding chapters examine the different challenges to decarbonisation in India. A summary of the chapters is provided in the subsequent sections.


\subsubsection{Chapter 2: Reducing stranded assets through early action in the Indian power sector}
The second chapter reveals that early climate action in the Indian power sector greatly reduces the risk of stranded assets under the Paris Agreement regime. Even assuming anticipated reductions in solar PV, wind, and battery costs, and current RE targets, model projections show that India continues to build coal power plants until 2030. In pathways where mitigation starts thereafter, to limit warming to well-below \SI{2}{\degreeCelsius}, almost half of this yet-to-be-built capacity gets stranded. If stringent climate mitigation is to start immediately, the stranding is lower, however this entails massive build up of RE and high carbon prices. 

A more politically feasible pathway in the near-term and an intermediate between the two policy scenarios, current policies and the \SI{2}{\degreeCelsius} consistent scenario, could be a `coal moratorium', where no power plants beyond those under construction are allowed and existing plants run until the end of their lifetime. Such a policy could have additional benefits --- e.g., keeping the plant load factor of existing coal plants at the current level, preventing the power system to get further locked into coal and thus necessitating large stranded assets in the future, opening the possibility of integrating emerging and cheaper power technologies in the future, and as mentioned before --— laying down the important groundwork for ambitious future climate policy. 

\subsubsection{Chapter 3: Climate policy accelerates structural changes in energy employment}
The third chapter uses the employment factor approach to estimate the number and structure of global jobs in the energy supply sector for two scenarios: NDC and \SI{1.5}{\degreeCelsius}-compatible. In the near-term, net gains are seen only in the \SI{1.5}{\degreeCelsius} policy case, despite relatively higher losses in the coal mining sector. For both these scenarios, the direct energy supply jobs decrease in the future (2050) compared to 2020, driven by improvements in labour productivity. However, higher climate policy ambition leads to higher employment. Lastly, the increase in cumulative solar and wind capacity, against the decrease in total fuel production, means that the O \& M jobs overtake fuel supply as the major share of total jobs in the future.
 
Despite the higher gains in jobs in the policy scenario, the regional picture could look very different. The extent to which employment considerations could affect the pace of decarbonisation depends on regional and skill-related factors e.g., size of the power sector, the overlap between regions of coal mines and RE-rich areas, skill-overlap between fossil and RE-jobs and other policies in place, e.g., encouraging local manufacturing, increasing investment in coal-related communities to help with the transition, etc.


\subsubsection{Chapter 4: Solar energy as an early just transition opportunity for coal-bearing states in India}
The fourth chapter argues that just transition needs to be adequately addressed in India's energy transition and that solar could play a limited but important role in bringing about the just transition. In India, both existing and planned solar and wind plants are concentrated in the south and western parts of India and overlap largely with areas of high resource potentials. On the other hand, fossil fuel infrastructure --- mainly coal mines and without power plants are located in the eastern states and would house future under-construction/planned fossil infrastructure. We show that the pathways based on current policies would create a sharp regional divide in future energy infrastructure and associated jobs. This would become even larger in decarbonisation scenarios, as it leads to higher jobs losses in coal mining in the eastern states but significant increases in jobs along the renewable value chain in the southern and western states. Without specific policies addressing this mismatch, the road towards decarbonisation could face political push back. Substituting all lost coal jobs with long-term jobs in solar energy would be extremely difficult as it would require a significant share of future all-India solar capacity additions to go to a few coal states.

At the same time dedicated policies to ensure early geographic diversification of solar locations could be an important policy component to help build broad support for the energy transition that is required for climate targets and could give India important benefits in terms of avoided climate impacts and local health. 


\section{Discussion and Policy implications}
In chapter \ref{ch:ERLPaper} we identified that an early move to stringent mitigation policies in India, for example a coal moratorium, could lessen the risk of future stranded assets of coal power plants. These plants, with their long life times create path-dependence, thereby affecting the future evolution of the power system. India should instead aim to further increase its renewable energy ambition. This might lead to near-term increase in price of electricity, however over a longer period proves to be cheaper than a fossil-dominant system \citep{luIndia2020}. Moreover, given the extremely dynamic nature of the current world renewable energy system with its steadily decreasing capital costs, India would do well not to lock into a system of coal power. There are now some indications that a moratorium on building no new coal power might emerge from existing circumstances in the power sector. The states of Gujarat, Chhattisgarh, and Karnataka  announced no new construction of coal power plants \citep{carboncopy2019}. National Thermal Power Corporation, India's largest coal-power generator, also announced that it will not pursue any new greenfield development of coal-fired power projects and plans to have 45\% of its installed capacity from RE by 2032  \citep{ians2020}. Furthermore, no new Indian coal-fired power plants were announced in 2019--2020 and there has been no progress on the 29.3 GW of pre-construction (includes announced, pre-permitted, and permitted plants) project pipeline during this time \citep{gem2021}. While the private sector is moving away from coal power, new plans and announcements mainly come from central and state government companies, financed primarily using central government funds \citep{shah2021}.

Although ambitious near-term policies for renewable energy and a moratorium on coal might have advantages as described above, they will shift the distribution of energy infrastructure and associated employment, as analysed in chapter \ref{ch:EnPolPaper}. For example, energy investment and jobs will shift away from regions with fossil resources to manufacturing centers of wind and solar components and areas of high solar and wind potentials. In fact, one reason for the government of India's continued push on coal is that an immediate and quick transition to renewable energy would entail significant imports and loss of local jobs. Although commissioning and installations jobs are created locally, the majority of the manufacturing of solar cells and panels takes place outside India. In 2018, domestic manufacturers had a market share of 7\% \citep{singh} in solar cells, with the majority of the imports from China \footnote{India fares much better in manufacture of wind energy components, with almost 70-80\% indigenisation \citep{ministryofnewandrenewableenergy2021a}}, implying that India would have to pay significant foreign exchange and transform from a mainly in-house manufacture, construction, and production of coal and coal power, to an import-based RE energy sector. India already depends significantly on imports for oil and gas, 75\% and 50\% respectively in 2019 \citep{iea2021}; coal with its large reserves, and coal power with an established legacy are seen as symbols of reliability, efficiency, and security \citep{montrone2021}. Although local solar cell and module manufacture failed to grow in India in the past, out-competed by its Chinese counterparts \citep{irena2019}, the government remains keen on developing an indigenous manufacturing base. After releasing tenders with locally manufactured components failed, it has recently brought up a production-linked incentive scheme to bolster domestic manufacture (of both panels and batteries) and deterred imports by increasing the customs duty on imported panels \citep{carboncopy2021,carboncopy2021a,carboncopy2021b}. Although domestic manufacture would create local jobs, they would possibly increase capital costs of solar, at least in the near-term. It remains to be seen how this would affect India's solar ambitions and in turn the pace of decarbonisation, and if the scheme proves to be more successful than its predecessors.

Chapter \ref{ch:ERLPaper2} begins with the premise that decarbonisation, both near and long-term, could be hindered if losers from the transition, be it states, firms, and the people employed are not adequately compensated or provided with alternate opportunities. In other words, an energy transition as mentioned in chapter \ref{ch:ERLPaper} might be feasible from a techno-economic point of view but might be politically infeasible. Chapter \ref{ch:ERLPaper2} therefore explores this feasibility by examining the current and future course of energy infrastructure and energy jobs concentration in India. In contrast to chapter \ref{ch:EnPolPaper}, where the focus is global and across countries, this chapter focuses on states within the country. The eastern states of Chhatisgarh, Odisha, and Jharkhand produce almost 60\% of coal in India, the mining bringing significant revenue for both the state and the central government. Although the people directly employed in coal mining are a very small fraction of the total workforce in these states, many people are informally employed in coal mining, and others indirectly or induced \footnote{induced jobs are created through consumption of goods and services of people employed directly or indirectly with an industry or sector} employed through coal mining - e.g., employed in temporary construction works in mining premises, truckers transporting mined coal and overburden, as staff in hospitals and schools run by coal companies, or in hotels and restaurants around coal mines \citep{bhushan2020} \footnote{this effect, however, maybe be hyper-local and the dependence of coal on total income fades with distance \citep{bhushan2020}.}. Coal mining also provides jobs in allied sectors, such as coal washeries. On the other hand, the wealthier western and southern states of India are projected to gain most of the energy transition, mainly due to high resource potentials in these states. Thus, without adequate compensation or work opportunities to replace coal, people could oppose the transition. In fact, according to \citet{bhushan2020}, one reason why centre-controlled Coal India Limited continues to run so many small-scale unprofitable mines is because of the pressure of workers' unions. 

\subsection{Future international climate action}
The three core chapters in this thesis have been primarily addressed to local governments, identifying barriers to decarbonisation and solutions how to overcome them. However, as mentioned in the introductory chapter, national action on climate is heavily influenced by action of other countries and agreements in international negotiations. Developing countries, especially India, have iterated consistently for decades that energy justice considerations (`common but differentiated responsibility') need to be recognized by developed nations so as not to stifle its development and that its climate commitments should not be seen under the same umbrella as theirs. It is clear that if achieving the Paris Agreement requires carbon neutrality by 2050, then based on differentiated responsibility, either richer nations need to take on more emission reductions or provide financial and technological support to developing countries to scale up their renewable capacity. However, so far developed nations have even fell short of the commitment of \$100 billion dollars a year by 2020 of climate finance. The COP 26 in Glasgow should therefore focus not only on increased mitigation effort of richer nations `at home' but action on providing cheap finance and technological know-how, for example of manufacturing of renewable energy components, to support developing countries build new renewable energy plants instead of new coal power plants and get locked into coal.

\section{Limitations}
Although the thesis uses a conceptual framework to investigate the barriers in power sector decarbonisation, the papers in this work do not cover every aspect of the framework. The second and third chapters cover the impact of mitigation scenarios on employment but they miss on other indicators affecting the political and institutional constraints of these scenarios. Chapter \ref{ch:ERLPaper}, for example, assumes that carbon pricing becomes immediately available in an economy and that the price could be set to be very high within a span of a few years. Experience from other countries has shown that a properly functioning carbon market, encompassing multiple sectors can take years and is often preceded by a long term of technology-specific targets and policies toward decarbonisation. The chapter also does not consider the distributional implications of applying a carbon revenue evenly across all polluting sectors. Here too, studies reveal that both the level of carbon price and how the revenue is used by the governments plays an important role in people's acceptance of mitigation measures. The chapter also assumes that competition between various technologies producing electricity happens on the power market. In reality, power distribution companies are often limited in their options due to long-term contracts with generators. Some of them even specify the payment of fixed charges even if no electricity is purchased, thus facing different prices compared to the wholesale price. Lastly, although most energy models include some representation of the rate of renewable capacity expansion in the near-term, for example through additional mark-up costs, these might not capture the full range of country-specific barriers in accessing finance and the risk of investments. In India, state-run distribution companies are under constant financial losses and owe billions of dollars to generators; in turn affecting the latter's investment returns. 

Chapter \ref{ch:EnPolPaper} and chapter \ref{ch:ERLPaper2} focus on the employment implications of mitigation scenarios of different stringency, thereby assessing how changes in employment of various technologies and stages of production could impact the pace of decarbonisation and what could be possible solutions to overcome these barriers. Due to the limitations of the methodology, we cannot comment on how the scenarios would impact people employed indirectly in the power sector or employed informally. These factors could become very important in areas immediate to where a resource is located. Chapter \ref{ch:ERLPaper2} also does not discuss on the political impact of the energy transition through the sectors affiliated to coal extraction. Coal accounts for almost half of both the total freight and the total freight revenue for the railways \citep{kamboj2018,iea2021}, and 60\% of the total coal consumed by the power plants is transported through rail. Train freight charges in India are some of the highest in the world, particularly because they in turn subsidise passenger transport. 

\section{Suggestions for future research}
Future research in context of the research theme should be directed into addressing the various limitations described in the last section. Firstly, pathways of energy transition, especially for low-income countries with high development priorities, need to provide a more holistic picture of the implications of the transition, so that the discussion is not reduced to one single parameter like the `cost of electricity' or `carbon price'. This means getting better at integrating key environmental externalities like air pollution but also social implications like inequality, hunger, distributional impacts, etc. which affect the political feasibility of mitigation scenarios. At the same time, care should be taken while extending energy and climate-energy models to beyond what they were originally conceived, for example, by encouraging multiple methods, approaches, and model intercomparison  exercises. Secondly, to increase the confidence of using the employment-factor approach for employment estimation, long-term empirical studies should be carried out on finding the employment factor along the value chain, especially in non-OECD countries, where data is currently scarce. Lastly, for the just transition of coal workers and communities in India, research should investigate opportunities beyond the energy sector.   

\section{Conclusion}
Fulfilling the goals of the Paris Agreement requires that all nations collectively pursue deep decarbonisation. The starting point in this endeavour is the power sector. Even for low-income countries, with high development priorities and low historic responsibilities, there are strong reasons to start decarbonizing the power sector now. However, there are barriers to this process which go beyond techno-economics, into governance, and institutional and political factors. Investigating the nature of these factors is essential to decarbonise quickly and effectively. In the preceding chapters we identified some of these barriers and proposed solutions how to overcome them. In chapter \ref{ch:ERLPaper}, the barriers took the form of stranded assets which could be overcome through increased near-term investments in renewable and a moratorium on construction of new coal plants. In chapter \ref{ch:EnPolPaper}, the barriers were structural changes to energy employment, specifically the losses in coal mining, which however could be reduced through job-transfer in the renewable energy sector, provided that retraining efforts are successful. In chapter \ref{ch:ERLPaper2}, the barriers became more nuanced as we compared how energy infrastructure would evolve regionally across India and found that solar could play a small but important role in the just transition of coal-bearing states.




\biblio
\end{document}