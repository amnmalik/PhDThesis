\documentclass[../thesis.tex]{subfiles}
\graphicspath{{\subfix{../pictures/}}}
\begin{document}
\chapter*{Abstract}
The role of developing countries like India in climate action has undergone a shift in the last five to ten years. Several factors have led to this development. Firstly, with the signing of the Paris Agreement and its emphasis on bottom-pledges, all countries have become co-enactors to mitigation. Secondly, continued scientific research on co-benefits and climate damages has reduced the gap between mitigation and development priorities. Lastly, capital costs of renewable energy (RE) have plummeted making them cheaper than new coal plants in most countries, thereby providing a solid economic incentive to increase the share of RE. Despite these developments, decarbonisation of the power sector in low-income countries faces significant socio-economic and political barriers. This dissertation identifies some of those barriers, eventually suggesting policy solutions to overcome them. While one publication of this cumulative dissertation has a global scope, the other two papers focus on India, a country with low cumulative historic emissions, but is currently the third-largest emitter of greenhouse gases (GHG). Per-capita energy consumption is still low, but it has one of the fastest growing electricity markets in the world. Thus, the policy decisions in the power sector in India can substantially affect the global goal to decarbonisation.

The first publication identifies the risk of carbon lock-ins in the power sector if India were to continue a trajectory based on current policies. We find that continued investment into fossils could eventually lead to stranded assets in the future because of the faster pace of decarbonisation required in scenarios achieving the Paris Agreement goals. Since most of the stranding arises from plants yet to be built, it can be avoided through additional capacity installations of RE, i.e., increasing current ambition in RE-deployment and limiting new coal power plants to those under construction. Most of the additional capacity would come from solar and wind, given their large resource potentials and favourable economic viability in India. The expansion potential of other sources like gas, nuclear, and hydro remains low, owing to constraints on supply, cost, and construction duration.

The second article compares different mitigation scenarios and analyses, on a global level but based on country-specific data, the labor market implications of a decarbonisation policies. Although ambitious policies supporting RE and discouraging coal power, e.g., through a coal moratorium, discussed above are favourable for (future) deep decarbonisation, they could lead to disruptive changes adversely affecting the employment situation, specifically the drastic losses in the fossil sector. We show that in the near-term, stringent mitigation results in a net increase in jobs compared to a weaker climate action scenario (based on currently pledged country objectives), mainly through gains in solar and wind jobs in construction, installation, and manufacturing, despite significantly higher losses in coal fuel supply. However, global energy jobs eventually peak, because the falling labour intensity (i.e. jobs per megawatt, due to increasing productivity) outpace increases in RE installations. In the future, total jobs are still higher in stringent mitigation  than in a scenario with less mitigation with most people employed in the operation and maintenance of RE infrastructure, unlike fuel extraction today.  Although stricter mitigation could lead to higher jobs globally, the role of employment in decarbonisation in specific regions could play out very differently. In countries with significant people employed in fossil-fuel industries, a just transition for those workers could become important. 

The third publication highlights that the regional mismatch of energy infrastructure in India could become a significant barrier to effective decarbonisation. Most of the coal mines and coal power plants in India are concentrated in the poorer eastern states of Chhattisgarh, Odisha, and Jharkhand, where it is an important source of both employment and public economy. On the other hand, the best RE potentials in India are concentrated in the relatively wealthier western and southern states and are home to current and planned RE installations.  Continued fossil investments in coal-bearing regions could widen this gap and in pathways to deep decarbonisation, strongly accelerate the loss of coal jobs. Without complementary opportunities, this would negatively impact the livelihood of people living in these areas. We show that dedicated policies to increase solar installations in coal regions could ensure early geographic diversification of solar energy. It could help build broad support for the energy transition, required for climate targets, and could give India important benefits in terms of avoided climate impacts and local health. At the same time, solar alone cannot provide a just transition and there is an urgent need for engagement with all stakeholders exploring challenges and other opportunities into the transition. 

In summary, despite the proliferation of climate considerations into decision-making at all political levels, there are still significant barriers to decarbonisation. Some of the most pressing challenges for fast-growing economies like India involve avoiding lock-ins in the power sector, which could have far-reaching consequences on the pace and cost of future decarbonisation. Higher-income nations could support the transition by providing cheaper RE-related finance and knowledge of increasing power system flexibility. At the same time, changes in the quantity and structure of jobs in the energy sector could also affect the pace of decarbonisation. Here, one key factor is the just transition of predominantly coal-bearing regions. The regional divide of fossil and RE assets and resources in India means that a regionally balanced transition from a fossil to a RE-based economy would not happen on its own; it needs dedicated policies supporting future solar installations in coal-bearing states. However, given the large size of the current coal workforce, additional solar capacity alone (in these regions) cannot replace all the lost jobs. It therefore requires to look for alternatives beyond the energy sector. 

\end{document}