\documentclass[../thesis.tex]{subfiles}
\graphicspath{{\subfix{../pictures/}}}

\begin{document}
\chapter*{Zusammenfassung}
Die Rolle von Entwicklungsländern wie Indien bei den Klimaschutzma\ss nahmen hat sich in den letzten fünf bis zehn Jahren gewandelt. Mehrere Faktoren haben zu dieser Entwicklung geführt. Erstens sind mit der Unterzeichnung des Pariser Abkommens und seiner Betonung der "`bottom-pledges"' % note: I changed the quotation marks. Maybe you need to add \selectlanguage{ngerman} and load a package like \usepackage{csquotes}
 alle Länder zu Mitakteuren beim Klimaschutz geworden. Zweitens hat sich durch wissenschaftliche Forschung über Klimaschäden und positive Nebeneffekte von Klimaschutz die Kluft zwischen Minderungs- und Entwicklungsprioritäten verringert. Drittens sind die Kapitalkosten für erneuerbare Energien (EE) drastisch gesunken, so dass sie in den meisten Ländern billiger sind als neue Kohlekraftwerke, was einen verlässlichen wirtschaftlichen Anreiz zur Erhöhung des Anteils regenerativer Energien bietet. Trotz dieser Entwicklungen sind die sozioökonomischen und politischen Hindernisse für die Dekarbonisierung des Stromsektors in Ländern mit niedrigem Einkommen erheblich. In dieser Dissertation werden einige dieser Hindernisse aufgezeigt und schlie\ss lich politische Lösungen zu deren Überwindung vorgeschlagen. Während eine Publikation dieser kumulativen Dissertation eine globale Perspektive einnimmt, konzentrieren sich die anderen beiden Artikel l auf Indien, dessen kumulierte historische Emissionen gering sind, das jedoch derzeit der drittgrö\ss te Emittent von Treibhausgasen (THG) ist. Der Pro-Kopf-Energieverbrauch ist immer noch niedrig, aber das Land hat einen der am schnellsten wachsenden Strommärkte der Welt. Daher können die politischen Entscheidungen im indischen Energiesektor das globale Ziel der Dekarbonisierung erheblich beeinflussen.

In der ersten Publikation wird das Risiko von Kohlenstoff-Lock-Ins im Energiesektor aufgezeigt, wenn Indien den auf der aktuellen Politik basierenden Kurs fortsetzen würde. Wir zeigen, dass ein Fortsetzen der Investitionen in fossile Energieträger in der Zukunft zu "`verlorenem Kapital"' (stranded assets) führt, sobald die Dekarbonisierung derart beschleunigt wird, dass die Ziele des Pariser Abkommens in den analysierten Szenarien erreicht werden. Da die meisten dieser Fehlinvestitionen aus noch zu bauenden Anlagen stammen, können sie vermieden werden, wenn zusätzliche EE-Kapazitäten aufgebaut und neue Kohlekraftwerke auf die im Bau befindlichen beschränkt werden. Der grö\ss te Teil der zusätzlichen Kapazität würde aus Sonnen- und Windenergie stammen, da sie über ein gro\ss es Potenzial verfügen und in Indien wirtschaftlich rentabel sind. Das Ausbaupotenzial anderer Energieträger wie Gas, Kernkraft und Wasserkraft bleibt aufgrund von Einschränkungen bei Angebot, Kosteneffizienz und Bauzeit gering.

Im zweiten Artikel werden verschiedene Minderungsszenarien verwendet % if you use compare in English, use "verglichen" in German instead of verwendet
 und auf globaler Ebene, aber auf der Grundlage länderspezifischer Daten, die Auswirkungen einer Dekarbonisierungspolitik auf den Arbeitsmarkt analysiert. Obwohl ehrgeizige politische Ma\ss nahmen zur Förderung von EE und zur Eindämmung der Kohleverstromung, z.B. durch ein Kohlemoratorium, wie oben erörtert für eine (künftige) tiefgreifende Dekarbonisierung günstig sind, könnten sie zu disruptiven Veränderungen führen, die sich nachteilig auf die Beschäftigungssituation auswirken, insbesondere durch drastische Verluste im fossilen Sektor. Wir zeigen, dass ein strenger Klimaschutz kurzfristig zu einem Nettozuwachs an Arbeitsplätzen im Vergleich zu einem schwächeren Klimaschutzszenario (basierend auf den derzeit zugesagten Länderzielen) führt, vor allem durch einen Zuwachs an Arbeitsplätzen in der Solar- und Windenergiebranche in den Bereichen Bau, Installation und Produktion, trotz deutlich höherer Arbeitsplatzverluste im Kohlesektor. Allerdings erreicht die Zahl der Arbeitsplätze im Energiesektor weltweit letztendlich ihren Höchststand, da die sinkende Arbeitsintensität (d. h. Arbeitsplätze/Megawatt, aufgrund steigender Produktivität) den Anstieg der EE-Installationen überkompensiert. In der Zukunft ist die Gesamtzahl der Arbeitsplätze bei schneller Dekarbonisierung immer noch höher als bei einem Szenario mit geringerem Klimaschutz, wobei die meisten Menschen nicht wie heute in der Brennstoffgewinnung, sondern im Betrieb und in der Wartung der EE-Infrastruktur beschäftigt sind. Obwohl strengerer Klimaschutz weltweit zu mehr Arbeitsplätzen führen könnte, könnten die Auswirkungen der Dekarbonisierung auf die Beschäftigung in einzelnen Regionen sehr unterschiedlich ausfallen. In Ländern, in denen viele Menschen in der Produktion fossiler Brennstoffe beschäftigt sind, könnte die Berücksichtigung sozialer Gerechtigkeit bei diesem Übergang im Sinne einer "`just transition"' wichtig werden.

Im dritten Artikel wird hervorgehoben, dass das regionale Ungleichgewicht der Energieinfrastruktur in Indien zu einem erheblichen Hindernis für eine wirksame Dekarbonisierung werden könnte. Die meisten Kohleminen und Kohlekraftwerke in Indien befinden sich in den ärmeren östlichen Bundesstaaten Chhattisgarh, Odisha und Jharkhand, wo sie eine wichtige Stütze des Arbeitsmarkts und der öffentlichen Wirtschaft darstellen. Andererseits konzentrieren sich die besten EE-Potenziale in Indien auf die wohlhabenderen westlichen und südlichen Bundesstaaten, in denen bestehende und geplante EE-Anlagen zu finden sind.  Fortgesetzte Investitionen in fossile Energieträger in den Kohleregionen könnten diese Kluft vergrö\ss ern und auf dem Weg zu einer tiefgreifenden Dekarbonisierung den Verlust von Arbeitsplätzen in der Kohleindustrie stark beschleunigen. Ohne Alternativmöglichkeiten würde sich dies negativ auf den Lebensunterhalt der in diesen Gebieten lebenden Menschen auswirken. Wir zeigen, dass gezielte Politikma\ss nahmen, um Solaranlagen in Kohleregionen zu installieren, eine frühzeitige geografische Diversifizierung der Solarenergie sicherstellen könnten. Dies könnte dazu beitragen, eine breite Unterstützung für die Energiewende aufzubauen, die für die Erreichung der Klimaziele erforderlich ist, und Indien wichtige Vorteile im Hinblick auf die lokale Gesundheit un die Vermeidung von Klimaschäden bringen. Gleichzeitig kann die Solarenergie allein keinen gerechten Übergang sicherstellen, und es besteht dringender Bedarf, alle Interessengruppen zu beteiligen, um Herausforderungen und weitere Möglichkeiten für diesen Übergang zu identifizieren.

Zusammenfassend gibt immer noch erhebliche Hindernisse für die Dekarbonisierung, obwohl Klimaaspekte bei der Entscheidungsfindung auf allen politischen Ebenen zunehmend berücksichtigt werden. Einige der dringendsten Herausforderungen für schnell wachsende Volkswirtschaften wie Indien bestehen darin, Lock-Ins im Energiesektor zu vermeiden, die weitreichende Folgen für das Tempo und die Kosten der künftigen Dekarbonisierung haben könnten. Länder mit höherem Einkommen könnten den Übergang unterstützen, in dem sie ihre Kenntnisse zur Erhöhung der Flexibilität des Stromsystems anbieten und für eine günstigere Finanzierung erneuerbarer Energien sorgen. Gleichzeitig könnten sich Veränderungen in der Anzahl und Struktur der Arbeitsplätze im Energiesektor auch auf das Tempo der Dekarbonisierung auswirken. Ein Schlüsselfaktor in diesem Zusammenhang ist ein gerechter Übergang in Regionen, in denen überwiegend Kohle gefördert wird. Die regionale Verteilung der fossilen und erneuerbaren Ressourcen in Indien bedeutet, dass ein regional ausgewogener Übergang von einer fossilen zu einer auf erneuerbaren Energien basierenden Wirtschaft nicht von alleine erfolgen würde; es bedarf spezieller politischer Ma\ss nahmen zur Unterstützung des Baus von Solaranlagen in den bisher kohlefördernden Bundesstaaten. In Anbetracht der gro\ss en Zahl der derzeit im Kohlebergbau Beschäftigten können zusätzliche Solarkapazitäten allein (in diesen Regionen) jedoch nicht alle verlorenen Arbeitsplätze ersetzen. Daher muss nach Alternativen au\ss erhalb des Energiesektors gesucht werden. 

\end{document}